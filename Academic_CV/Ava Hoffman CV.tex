% Written by Ava Hoffman
% Please use as you like, but it would be nice if you credited me :)
% 
% Use LaTeX to compile and ensure style file (cv.cls) is in the same directory.
%
%%%%%%%%%%%%%%%%%%%%%%%%%%%%%%%%%%%%%%%%%

%----------------------------------------------------------------------------------------
%	PREAMBLE
%----------------------------------------------------------------------------------------

\documentclass{cv}

\begin{document}

%----------------------------------------------------------------------------------------
%	PROFESSIONAL DATA
%----------------------------------------------------------------------------------------

\name{AVA M. HOFFMAN}

\section*{PROFESSIONAL DATA}

Fred Hutchinson Cancer Center

Department of Biostatistics

1100 Fairview Avenue North

Seattle, WA 98109

\emailcontact{ahoffma2@fredhutch.org} $\cdot$ \emailcontact{ava.hoffman@jhu.edu} $\cdot$ \emailcontact{avamariehoffman@gmail.com}

Pronouns: she/her/hers

\subsection*{Social Media}

github: avahoffman

website: \href{https://www.avahoffman.com}{https://www.avahoffman.com}

%----------------------------------------------------------------------------------------
%	EDUCATION AND TRAINING
%----------------------------------------------------------------------------------------

\section*{EDUCATION AND TRAINING}

Ph.D. / 2019 -- Colorado State University, Ecology

B.S. / 2012 -- University of Virginia, Biology \textit{with distinction}

\subsection*{Postdoctoral Training}

2020--2021 Department of Earth and Planetary Sciences,  Johns Hopkins University

%----------------------------------------------------------------------------------------
%	PROFESSIONAL EXPERIENCE
%----------------------------------------------------------------------------------------

\section*{PROFESSIONAL EXPERIENCE}

\subsection*{Fred Hutchinson Cancer Center}

Senior Staff Scientist, Department of Biostatistics (7/1/2022-present)

\subsection*{Johns Hopkins University}

Faculty Associate, Department of Biostatistics, Johns Hopkins Bloomberg School of Public Health (7/1/2022-present)

Research Associate, Department of Biostatistics, Johns Hopkins Bloomberg School of Public Health (5/1/21--6/30/2022)

Postdoctoral Fellow, Department of Earth and Planetary Sciences, Johns Hopkins Krieger School of Arts \& Sciences (3/23/20--4/30/21)

\subsection*{Other Professional Experience}

Data Scientist, Boston Consulting Group (3/4/19--3/4/20), \textit{data science-based consulting role working for corporate clients}

Data Science Fellow, Insight Data Science (9/10/18--3/4/19), \textit{applied learning role building both technical products and career-focused community among fellows}

Laboratory Assistant, Biotechnology Division, DuPont-Pioneer (2012--2013)

Undergraduate Researcher, Department of Biology, University of Virginia (2011--2012)

Research Intern, United States Forest Service at Coweeta Hydrologic Laboratory LTER (2011)

Laboratory Assistant, Department of Biology, Virginia Commonwealth University (2010)

%----------------------------------------------------------------------------------------
%	PROFESSIONAL ACTIVITIES
%----------------------------------------------------------------------------------------

\section*{PROFESSIONAL ACTIVITIES}

\subsection*{Society Memberships and Leadership}

R--Ladies Baltimore, member (2020--present)

500 Women Scientists, pod member (2016--present)

Ecological Society of America, member (2012--present)

Women in Machine Learning and Data Science (Boston), member (2019--2020)

Front Range Student Ecology Symposium, Colorado State University, abstract review committee (2015--2016)

Front Range Student Ecology Symposium, Colorado State University, executive committee (2014--2015)

Front Range Student Ecology Symposium, Colorado State University, webmaster (2014--2015)

Graduate Degree Program in Ecology, Colorado State University, Journal Club founder and member (2014)

%----------------------------------------------------------------------------------------
%	EDITORIAL AND OTHER PEER REVIEW ACTIVITIES
%----------------------------------------------------------------------------------------

\section*{EDITORIAL AND OTHER PEER REVIEW ACTIVITIES}

\subsection*{Peer Review}

Journals: Agronomy, Environmental and Experimental Botany, Global Change Biology, Journal of Ecology, New Phytologist, Plant Biology, Plant Ecology, Plants, PloS One

%----------------------------------------------------------------------------------------
%	HONORS AND AWARDS
%----------------------------------------------------------------------------------------

\section*{HONORS AND AWARDS}

\subsection*{Awards}

Johns Hopkins Bloomberg School of Public Health Excellence in Teaching (2024 Winter Institute, 2023 Summer Institute, 2023 Winter Institute, 2022 Summer Institute, 2022 Winter Institute)
	
National Science Foundation Postdoctoral Research Fellowship in Biology (3/9/21, declined) %$216,000

Colorado State University Research Mentoring to Advance Inclusivity in STEM Research Grant (1/1/18) %$2500

Colorado State University School of Global Environmental Sustainability Leadership Fellowship (9/1/17) 

United States Department of Agriculture NIFA-AFRI Predoctoral Fellowship (1/1/17) %$94,999

Colorado State University Graduate Degree Program in Ecology Travel Grant (4/4/16) %$175

Colorado State University Graduate Degree Program in Ecology Research Grant (4/2/16) %$2000

City of Boulder Colorado Open Space and Mountain Parks Research Grant (3/1/2016) %$5669

Colorado State University Vice President of Research Graduate Fellow (2/29/16) %$3125

The Nature Conservancy J.E. Weaver Competitive Grant (2/10/16) %$1000

Colorado State University Department of Biology Harold Harrington Fellowship (4/10/15) %$1900

Colorado State University Department of Biology Stavros Family Fund Scholarship (4/10/15) %3021

National Science Foundation Graduate Research Fellowship Program (Honorable mention, 3/31/14) 

Colorado State University Department of Biology Sharon E. and David E. Kabes Scholarship (2014) %$590

Colorado State University Department of Biology Graduate Fellowship Award (2013) %$1000

University of Virginia Undergraduate Research Travel Grant (2012) %$1000

%2016 & \textbf{The Ecological Society of America} $\cdot$ `Hackathon' beginner app developer first place award\\
%2014-2015 & \textbf{Colorado State University} $\cdot$ Department of Biology Travel Awards (\$2258) \\

%----------------------------------------------------------------------------------------
%	PUBLICATIONS
%----------------------------------------------------------------------------------------

\section*{PUBLICATIONS}

\subsection*{Journal Articles (Peer Reviewed)}

\begin{pubenum}

%\textbf{Hoffman, AM}, M Swall*, JA Bushey, TW Ocheltree, and MD Smith. Bouteloua Gene Expession

%\textbf{Hoffman, AM}, CC Hoffman, F Chaves, A Felton, JE Gray, W Mao, A Tatarko, L Vilonen, L Gerhardi, MD Smith. Predicting dryland ecosystem responses to global change through dominant grasses. \textit{Functional Ecology}.\\

%\textbf{Hoffman, AM}, NP Lemoine, TW Ocheltree. Metabolomic and physiological responses to nitrogen addition in dominant semi-arid grassland species. \textit{Plos One}.\\

\item Savonen, C, C Wright, \textbf{A Hoffman}, E Humphries, K Cox, F Tan, J Leek (2024) Motivation, inclusivity, and realism should drive data science education. \textit{F1000Research}. \doi{10.12688/f1000research.134655.2}

\item Humphries, EM, C Wright, \textbf{AM Hoffman}, C Savonen, JT Leek (2023) What's the best chatbot for me? Researchers put LLMs through their paces. \textit{Nature}. \doi{10.1038/d41586-023-03023-4 }

\item Bushey, JA, \textbf{AM Hoffman}, TW Ocheltree, S Gleason, MD Smith (2023) Water limitation reveals local adaptation and plasticity in the drought tolerance strategies of \textit{Bouteloua gracilis}. \textit{Ecosphere}. \doi{10.1002/ecs2.4335}

\item Savonen, C, C Wright, \textbf{AM Hoffman}, J Muschelli, K Cox, FJ Tan, JT Leek (2022) Open-source Tools for Training Resources -- OTTR. \textit{Journal of Statistics and Data Science Education}. \doi{10.1080/26939169.2022.2118646}

\item The Genomic Data Science Community Network (R Alcazar, M Alvarez, R Arnold, M Ayalew, LG Best, MC Campbell, K Chowdhury, KEL Cox, C Daulton, Y Deng, C Easter, K Fuller, S Tabassum Hakim, \textbf{AM Hoffman}\footnotemark[1], N Kucher, A Lee, J Lee, JT Leek, R Meller, LB Méndez, MP Méndez-González, S Mosher, M Nishiguchi, S Pratap, T Rolle, S Roy, R Saidi, MC Schatz, S Sen, J Sniezek, E Suarez Martinez, FJ Tan, J Vessio, K Watson, W Westbroek, J Wilcox, X Xie) (2022) Diversifying the Genomic Data Science Research Community. \textit{Genome Research}. \doi{10.1101/gr.276496.121}

\item Carroll, CJW, IJ Slette, RJ Griffin-Nolan, LE Baur, \textbf{AM Hoffman}, EM Denton, JE Gray, AK Post, MK Johnston, Q Yu, SL Collins, Y Luo, MD Smith, AK Knapp (2021) Is a drought a drought in grasslands? Productivity responses to different types of drought. \textit{Oecologia}. \doi{10.1007/s00442-020-04793-8}

\item\textbf{Hoffman, AM}\footnotemark[1] and MD Smith (2020) Nonlinear drought plasticity reveals intraspecific diversity in a dominant grass species. \textit{Functional Ecology}. \doi{10.1111/1365-2435.13713}\footnotemark[\value{footnote}]

\item Knapp, AK, A Chen, RJ Griffin-Nolan, LE Baur, CJW Carroll, JE Gray, \textbf{AM Hoffman}, X Li, AK Post, IJ Slette, SL Collins, Y Luo, MD Smith (2020) Resolving the Dust Bowl paradox of grassland responses to extreme drought. \textit{PNAS}. \doi{10.1073/pnas.1922030117}

\item\textbf{Hoffman, AM}\footnotemark[1], JA Bushey, TW Ocheltree, MD Smith (2020) Genetic and functional variation across regional and local scales is associated with climate in a foundational prairie grass. \textit{New Phytologist}. \doi{10.1111/nph.16547}

\item Wilcox, KR, SE Koerner, DL Hoover, AK Borkenhagen, DE Burkepile, SL Collins, \textbf{AM Hoffman}, KP Kirkman, AK Knapp, T Strydom, DI Thompson, and MD Smith (2020) Rapid recovery of ecosystem function following extreme drought in a South African savanna-grassland. \textit{Ecology}. \doi{10.1002/ecy.2983}

\item\textbf{Hoffman, AM}\footnotemark[1], H Perretta\footnotemark[2], NP Lemoine, and MD Smith (2019) Blue grama grass genotype affects palatability and preference by semi-arid steppe grasshoppers. \textit{Acta Oecologia}. \doi{10.1016/j.actao.2019.03.001}

\footnotetext[1]{Indicates corresponding author}
\footnotetext[2]{Indicates a mentored student}

\item Griffin-Nolan, RJ, D Blumenthal, S Collins, T Farkas, \textbf{AM Hoffman}, K Mueller, TW Ocheltree, MD Smith, AK Knapp (2019) Shifts in plant functional composition following long-term drought in grasslands. \textit{Journal of Ecology}. \doi{10.1111/1365-2745.13252}

\item Smith, MD, SE Koerner, AK Knapp, ML Avolio, FA Chaves, EM Denton, J Dietrich, DJ Gibson, J Gray, \textbf{AM Hoffman}, DL Hoover, KJ Komatsu, A Silletti, KR Wilcox, Q Yu, and JM Blair (2019) Mass ratio effects underlie ecosystem responses to environmental change. \textit{Journal of Ecology}. \doi{10.1111/1365-2745.13330}

\item\textbf{Hoffman, AM}\footnotemark[1] and MD Smith (2018) Thinking inside the box: Tissue culture for plant propagation in a key ecological species, \textit{Andropogon gerardii}. \textit{American Journal of Plant Sciences}. \doi{10.4236/ajps.2018.910144} 

\item Knapp, AK, C Carroll, RJ Griffin-Nolan, IJ Slette, F Chaves, L Baur, AJ Felton, JE Gray, \textbf{AM Hoffman}, NP Lemoine, W Mao, A Post, MD Smith (2018) A reality check for climate change experiments: do they reflect the real world? \textit{Ecology}. \doi{10.1002/ecy.2474}

\item Griffin-Nolan, RJ, JA Bushey, CJW Carroll, A Challis, J Chieppa, M Garbowski, \textbf{AM Hoffman}, AK Post, IJ Slette, D Spitzer, D Zambonini, TW Ocheltree, DT Tissue, AK Knapp  (2018) Trait selection and community weighting are key to understanding ecosystem responses to changing precipitation regimes. \textit{Functional Ecology}. \doi{10.1111/1365-2435.13135}

\item\textbf{Hoffman, AM}, ML Avolio, AK Knapp, MD Smith (2018) Co-dominant grasses differ in gene expression under experimental climate extremes in native tallgrass prairie. \textit{PeerJ}. \doi{10.7717/peerj.4394}

\item\textbf{Hoffman, AM}\footnotemark[1] and MD Smith (2017) Gene expression differs in codominant prairie grasses under drought. \textit{Molecular Ecology Resources}. \doi{10.1111/1755-0998.12733}

\item Avolio, ML, \textbf{AM Hoffman}, MD Smith (2017) Linking gene regulation, physiology, and plant biomass allocation in \textit{Andropogon gerardii} in response to drought. \textit{Plant Ecology}. \doi{10.1007/s11258-017-0773-3}

\item Lemoine, NP, \textbf{AM Hoffman}, A Felton, L Baur, F Chaves, J Gray, Q Yu, MD Smith (2016) Underappreciated problems of low statistical power in ecological field studies. \textit{Ecology}. \doi{10.1002/ecy.1506}

\item Smith, MD, \textbf{AM Hoffman}, ML Avolio (2016) Gene expression patterns of two dominant tallgrass prairie species differ in response to warming and altered precipitation. \textit{Scientific Reports}. \doi{10.1038/srep25522}

\item Mellor, KE, \textbf{AM Hoffman}, MP Timko (2012) Use of ex vitro composite plants to study the interaction of cowpea (\textit{Vigna unguiculata} L.) with the root parasitic angiosperm \textit{Striga gesnerioides}. \textit{Plant Methods}. \doi{10.1186/1746-4811-8-22}

\sloppy % break url onto next line
\item\textbf{Hoffman, AM}\footnotemark[1] (2012) Estimating tree transpiration accurately depends on wood type and species: a study of four southern Appalachian tree species. \textit{The Oculus}. \href{https://issuu.com/theoculus/docs/spring2012}{https://issuu.com/theoculus/docs/spring2012}

\item Zinnert, JC, JD Nelson, \textbf{AM Hoffman} (2011) Effects of salinity on physiological responses and the photochemical reflectance index in two co-occurring coastal shrubs. \textit{Plant \& Soil}. \doi{10.1007/s11104-011-0955-z}

\end{pubenum}

\subsection*{Articles, Editorials and Other Publications Not Peer Reviewed}

\textbf{Hoffman, AM}, C Wright (2024) Ten simple rules for teaching introduction to R. \textit{EdArXiv}. \doi{10.35542/osf.io/g45vz}

Afiaz, A, AA Ivanov, J Chamberlin, D Hanauer, CL Savonen, MJ Goldman, M Morgan, M Reich, A Getka, A Holmes, S Pati, D Knight, PC Boutros, S Bakas, JG Caporaso, G Del Fiol, H Hochheiser, B Haas, PD Schloss, JA Eddy, J Albrecht, A Fedorov, L Waldron, \textbf{AM Hoffman}, RL Bradshaw, JT Leek, C Wright (2024) Evaluation of software impact designed for biomedical research: Are we measuring what's meaningful? \textit{ArXiv}. \doi{10.48550/arXiv.2306.03255}

Zinnert, JC, JD Nelson, JK Vick, \textbf{AM Hoffman}, DR Young (2010) Rethinking chlorophyll responses to stress: Fluorescence and reflectance remote sensing in a coastal environment. Proceedings of the 4th International Workshop on Remote Sensing of Vegetation Fluorescence, Valencia, Spain.

%----------------------------------------------------------------------------------------
%	PRACTICE ACTIVITIES
%----------------------------------------------------------------------------------------

\section*{PRACTICE ACTIVITIES}

\subsection*{Media Coverage}

Arnold, C. How Biased Data and Algorithms Can Harm Health: Public health researchers are working to uncover and correct unfairness in AI (2022) \textit{Hopkins Bloomberg Public Health magazine}. [\href{https://magazine.publichealth.jhu.edu/2022/how-biased-data-and-algorithms-can-harm-health}{Available online}]

Jensen, B. Research That's In the Weeds (2020) \textit{Johns Hopkins Magazine}. [\href{https://hub.jhu.edu/magazine/2020/winter/ava-hoffman-urban-ecology/}{Available online}]

\subsection*{Presentations to Policymakers, Communities, and Other Stakeholders}

Front Range Open Space Research Symposium (Boulder, CO, 4/11/17): Phenotypic diversity within dominant blue grama grass across an aridity gradient

%----------------------------------------------------------------------------------------
%	SOFTWARE
%----------------------------------------------------------------------------------------

\section*{SOFTWARE AND TEMPLATES}

\subsection*{GitHub Templates}

Online Tools for Training Resources (OTTR) Template - \textit{A GitHub template that simplifies and accelerates publishing course content in bookdown format or to Leanpub and Coursera. Created with Candace Savonen, Carrie Wright, and others.} [\href{https://github.com/jhudsl/OTTR_Template}{Available on GitHub}]

AnVIL Template - \textit{A GitHub template variation of the OTTR Template that automatically formats and generates content specific to the AnVIL Project. Created with Katherine Cox.} [\href{https://github.com/jhudsl/AnVIL_Template}{Available on GitHub}]

\subsection*{Software \& Tools}

DaSEH website portal - \textit{A website for participants to learn more about and download materials associated with the Data Science for Environmental Health NIH course.} [\href{https://daseh.org}{Available at daseh.org}]

BioDIGSData - \textit{the R Data Package for the BioDIGS Project.} [\href{https://github.com/fhdsl/BioDIGSData}{Available on GitHub}]

BioDIGS website and data portal - \textit{A website for students and faculty to learn more about and access data from the BioDIGS project.} [\href{https://biodigs.org}{Available at biodigs.org}]

DMS Helper - \textit{A text tool for helping Fred Hutch researchers and others create their NIH Data Sharing Plan.} [\href{https://dmshelper.fredhutch.org}{Available at dmshelper.fredhutch.org}]

AnVIL Collection - \textit{An automatically generating resource documenting all completed AnVIL and GDSCN content.} [\href{https://github.com/fhdsl/AnVIL_Collection}{Available on GitHub}]

DaSL Collection - \textit{An automatically generating resource documenting all completed Data Science Lab content.} [\href{https://github.com/fhdsl/DaSL_Collection}{Available on GitHub}]

Fred Hutch Letterhead - \textit{A LaTeX template for Fred Hutch-themed letterhead.} [\href{https://github.com/fhdsl/FH_letterhead}{Available on GitHub}]


%----------------------------------------------------------------------------------------
% Break page and add Part II header

\newpage
\name{AVA M. HOFFMAN}
\parttwo

%----------------------------------------------------------------------------------------

%----------------------------------------------------------------------------------------
%	TEACHING
%----------------------------------------------------------------------------------------

\section*{TEACHING}


\subsection*{Capstone Advisees}

Robotham, Daniel J., B.S. in Biological Sciences with Honors, Colorado State University (presented 5/15/15)

\hangpara{5pt}{0}Thesis Title: Determining the effects of water stress on co-occurring native \textit{Andropogon gerardii} and exotic (\textit{Bothriochloa bladhii}) C4 grasses

\subsection*{Research Advisees}

\subsubsection*{Postdoctoral Fellows}

Isaac, Katherine, Fred Hutchinson Cancer Center (2023--present)

Humphries, Elizabeth, Johns Hopkins University and Fred Hutchinson Cancer Center (2022--2024)

Nwigwe, Ifeoma, Johns Hopkins University (2022--present)

\subsubsection*{Master's Students}

Nwigwe, Ifeoma, Master's of Public Health Student Research Assistant, Johns Hopkins University (2022)

Alaku, Chinemeihe, Master's of Public Health Student Research Assistant, Johns Hopkins University (2022)

Zaman, Fatima, Master's of Public Health Student Research Assistant, Johns Hopkins University (2022)

\subsubsection*{Undergraduate Students}

Jawara, Jainaba, Summer Research Intern, Fred Hutch Cancer Center (2023)

Rodriguez, Natalie, Research Assistant, Johns Hopkins University (2021)

Swall, Madeleine, Student Researcher and Research Mentoring to Advance Inclusivity in STEM mentee, Colorado State University (2018)

Perretta, Holly, REU Student Researcher, Colorado State University (2016--2017)

Lock, Abigail, Research Assistant, Colorado State University (2017)

Gaudrealt, Brigitte, Research Assistant, Colorado State University (2016)

Magbual, Brianna, Research Assistant, Colorado State University (2014--2015)

Brown, Destiny, Research Assistant, Colorado State University (2014)

\subsection*{Classroom Instruction -- Instructor of Record}

\subsubsection*{Johns Hopkins University}

\href{https://jhudatascience.org/intro_to_R_class/}{Introduction to \texttt{R} for Public Health Researchers}

\begin{itemize}

\item 140.604.73 -- Winter Institute 2024, enrollment: 28

\item 140.604.79 -- Summer Institute 2023, enrollment: 40

\item 140.604.73 -- Winter Institute 2023, enrollment: 23

\item 140.604.79  -- Summer Institute 2022, enrollment: 41

\item 140.604.73 -- Winter Institute 2022, enrollment: 27

\item 140.604.11 -- Summer Institute 2021, enrollment: 33

\end{itemize}

\href{https://jhudatascience.org/Baltimore_Community_Course/index.html}{Baltimore Community Data Science}, an interdisciplinary Special Topics Course

\begin{itemize}

\item 140.801.01 -- Fall 2023, enrollment: 9

\item 140.604.73 -- Spring 2022, enrollment: 11

\end{itemize}

\subsection*{Classroom Instruction -- Teaching Fellow / Assistant}

\subsubsection*{Colorado State University}

Molecular and General Genetics (BZ 350), Recitation instructor (Fall 2016)

Community Ecology (ECOL 600), Recitation instructor (Spring 2016)

Foundations of Ecology (ECOL 505), Recitation instructor with class lectures (Fall 2015)

Molecular and General Genetics (BZ 350), Recitation instructor (Spring 2015)

Molecular and General Genetics (BZ 350), Recitation instructor (Fall 2014)

\subsubsection*{University of Virginia}

Organic Chemistry Lab II (CHEM 2421), Laboratory instructor (Spring 2011)

Organic Chemistry Lab I (CHEM 2411), Laboratory instructor (Fall 2010)

\subsection*{Short Courses}

\href{https://hutchdatascience.org/SeattleStatSummer_R}{Seattle Stat Summer R Training}, Four-day training series for Fred Hutch Seattle Stat Summer Interns (2023)

\href{http://sisbid.github.io/Data-Wrangling/}{Data Wrangling with R}, University of Washington Summer Institute (2023, 2022, 2021)

\subsection*{Other Teaching}

Introduction to AnVIL at Fred Hutch Cancer Center: In-person workshop (2024)

NHGRI Research Training and Career Development Annual Meeting: Introduction to AnVIL: In-person workshop (2024)

AnVIL Demos: Collecting and subsetting SRA data: Online webinar and Q\&A (2023)

Data Management \& Sharing Plans for NIH Proposals: In-person workshop at Fred Hutch Cancer Center (2023)

AnVIL Demos: Epigenetic Data Analysis on AnVIL: Online webinar and Q\&A (2023)

ASHG Virtual Workshop: Satisfying NIH data sharing and management requirements with Terra and AnVIL (2023)

AnVIL Demos: Single-cell Analysis with RStudio \& Bioconductor: In-person workshop: NIH NHGRI Intramural Research Group (2023)

What is AnVIL?: Online training: NIH NHGRI Extramural Research Group (2023)

Tools for Applied Data Science Using Cloud-Based Platforms: Online workshop, Virtual Applied Data Science Training Institute (2023)

WDL 101: Using WDL Workflows on AnVIL: Online workshop (2022)

GDSCN Train the Trainer: SARS-CoV-2 on Galaxy: Online workshop (2022)

Data Visualization using R and ggplot: Guest lecture, Colorado State University (2016)

\subsection*{Educational Resources}

\subsubsection*{Fred Hutch Data Science Lab (DaSL)}

Created course materials for the research community: 

\begin{itemize}

\item \href{https://hutchdatascience.org/AI_for_Decision_Makers/}{AI for Decision Makers}: A specialization of several courses intended for executives, decision-makers, and business leaders across industries to help them understand the strategic applications of AI and machine learning. Created with Carrie Wright, Candace Savonen, Elizabeth Humphries, and others.

\item \href{https://hutchdatascience.org/AI_for_Efficient_Programming/}{AI for Efficient Programming}: A course on using large language models (LLMs) as part of an efficient computer programming practice. Created with Carrie Wright, Candace Savonen, Elizabeth Humphries, and others.

\item \href{https://hutchdatascience.org/NIH_Data_Sharing/}{NIH Data Management and Sharing Policy Course}: A course outlining new policy requires, places to share data, and how to deal with possible challenges associated with the policy. Created with Carrie Wright and others.

\item \href{https://hutchdatascience.org/FH_Cluster_Guide/}{Fred Hutch Cluster 101}: A short course to get Fred Hutch researchers running on the Fred Hutch cluster quickly and efficiently.

\item \href{https://hutchdatascience.org/Using_Leanpub/}{Using Leanpub}: An introduction to navigating courses on Leanpub as a student or as an instructor.

\end{itemize}

\subsubsection*{Analysis Visualization and Informatics Lab-space (AnVIL) Project}

Created course materials for users to understand and better leverage the \href{https://anvilproject.org/}{AnVIL cloud computing platform} for education and research, including: 

\begin{itemize}

\item \href{https://hutchdatascience.org/AnVIL_Demos/}{AnVIL Demos}: A series of small, bite-sized lessons suitable for a new AnVIL user, including an introduction to AnVIL, exploration of WDL workflows, and a Bioconductor analysis. Created with Frederick Tan, Stephen Mosher, Natalie Kucher, and others.

\item \href{https://hutchdatascience.org/AnVIL_Data_Subsetting}{AnVIL: Subsetting Your Data with WDL}: A course that helps AnVIL users subset large genomic data for debugging and testing.

\item \href{https://hutchdatascience.org/AnVIL_SRA_Data/}{AnVIL: SRA Data}: A guide for bringing Sequence Read Archive (SRA) data into AnVIL.

\item \href{https://hutchdatascience.org/AnVIL_Book_Epigenetics_Intro/}{AnVIL Epigenetics Introduction}: An introduction to analysis of epigenetic data and epigenetics concepts on AnVIL. Created with Ifeoma Nwigwe.

\item \href{https://www.youtube.com/watch?v=tVh93e6TzCE\&list=PL6aYJ_0zJ4uCABkMngSYjPo_3c-nUUmio}{AnVIL Shorts: Learn about AnVIL in 2 Minutes}: a series of two-minute videos for new users to quickly understand multiple concepts and personas on AnVIL.

\item \href{https://jhudatascience.org/AnVIL_Book_Getting_Started/}{Getting Started on AnVIL}: a series of step-by-step guides for setting up accounts focused on three personas: PIs, Analysts, and Consortia. Also includes custom videos created using JHU software packages.

\item \href{https://jhudatascience.org/AnVIL_Book_Instructor_Guide/}{AnVIL Instructor Guide}: a guide to help classroom instructors who are new to AnVIL set up their accounts and start developing content.

\end{itemize}

\subsubsection*{Genomic Data Science Community Network (GDSCN) Project}

Created course materials for \href{https://www.gdscn.org/}{GDSCN} faculty to use in their classrooms, including: 

\begin{itemize}

\item \href{https://hutchdatascience.org/GDSCN_BioDIGS_Book/}{GDSCN Book: BioDIGS in the Classroom}: a companion training guide for BioDIGS, a GDSCN project that brings a research experience into the classroom.This guide focuses on training for handling soil property data on the AnVIL platform. Created with Elizabeth Humphries.

\item \href{https://jhudatascience.org/GDSCN_Book_Statistics_for_Genomics_Differential_Expression/}{Statistics for Genomics: Differential Expression}: A set of lab modules in the R programming language for an introduction to differential gene expression.

\item \href{https://jhudatascience.org/GDSCN_Book_SARS_Galaxy_on_AnVIL/}{GDSCN Book: SARS with Galaxy on AnVIL}: a series of resources for instructors to engage students in a cloud-based Galaxy activity on AnVIL, focused on SARS-CoV-2 variant detection. Includes instructional videos, background lectures for instructors to use in their own classes, and a student lab activity guide.

\item \href{https://anvil.terra.bio/#workspaces/gdscn-exercises/SARS-CoV-2-Genome}{AnVIL Workspace} for GDSCN members to use to follow the SARS-CoV-2 variant activity.

\end{itemize}

%----------------------------------------------------------------------------------------
%	RESEARCH GRANT PARTICIPATION
%----------------------------------------------------------------------------------------

\section*{RESEARCH GRANT PARTICIPATION}

\subsection*{Current Support}

Project Title: Expanding the AnVIL (Analysis, Visualization, and Informatics Lab-space) (\href{https://reporter.nih.gov/search/n8pPuYzcskWEltb40MVBkA/project-details/10748042}{U24HG010263-07})

Dates: 09/21/2018--04/30/2028 (renewed in limited competition, 2023)

Sponsoring Agency: National Institutes of Health (NHGRI)

Principal Investigator: Michael C. Schatz

Main Grant Objective: Create user-centered solutions for data access, analysis, and visualization that enable investigators across all levels of expertise to fully utilize genomic datasets while using existing tools ported to a powerful cloud based platform

Role: MPI

\vspace{5mm}

Project Title: Short Course in Data Science for Environmental Public Health (\href{https://reporter.nih.gov/search/n8pPuYzcskWEltb40MVBkA/project-details/10746327}{5R25ES035590-02})

Dates: 08/25/2023--07/31/2027

Sponsoring Agency: National Institutes of Health (NIEHS)

Principal Investigator: Ava M. Hoffman

Main Grant Objective: Create a more diverse environmental health community trained in data science principles through the creation of a short course that combines online learning and an in-person project-focused intensive.

Role: PI

\vspace{5mm}

Project Title: Expanding the Genomic Data Science Community Network for NHGRI (\href{https://reporter.nih.gov/search/hBYvgxL2yUO5iw9lGMoYTA/project-details/10944109}{75N92023P00302-0-0-1})

Dates: 09/01/2023--08/31/2025

Sponsoring Agency: National Institutes of Health (NHGRI)

Principal Investigator: Michael C. Schatz

Main Grant Objective: Support the development of training materials by faculty at institutions with fewer resources and increase dissemination and outreach activities, enabling a broader spectrum of institutions to have education and research access to genomic data science through the Genomic Data Science Community Network.

Role: Project Lead

\subsection*{Past Support}

Project Title: EVO-LTER: Leveraging long-term ecological research in grasslands: facilitating collaborations between ecologists and evolutionary biologists (DEB-2110351)

Dates: 3/2022--2/2024

Sponsoring Agency: National Science Foundation

Principal Investigator: Meghan L. Avolio

Funding Amount: \$99,933

Main Grant Objective: Facilitate collaborative workshop to foster more evolutionary and molecular work at NSF Long Term Ecological Research sites

Role: Co-PI

\vspace{5mm}

Project Title: The Genomic Data Science Community Network (NIH/NHLBI 75N92020P00235)

Dates: 9/2020--8/2022

Sponsoring Agency: National Institutes of Health

Principal Investigator: Michael C. Schatz

Main Grant Objective: Facilitate the formation and empowerment of a collaborative network of faculty from historically underserved universities and colleges in genomic data science education and research

Role: Content Developer

\vspace{5mm}

Project Title: Collaborative Research: MSB-FRA: Alternative futures for the American residential macro systems (DEB-1836034)

Dates: 01/2017--12/2020

Sponsoring Agency: National Science Foundation

Principal Investigator: Meghan L. Avolio

Funding Amount: \$90,963

Main Grant Objective: Reveal genomic population structure and phenotypic differences among urban weeds in North American cities

Role: Postdoctoral fellow / Project Manager

\vspace{5mm}

Project Title: CityCress: A plant model for genetic, phenotypic, and gene regulatory relationships in urban environments (NSF-2109727)

Dates: Award declined by PI

Sponsoring Agency: National Science Foundation

Principal Investigator: Ava M. Hoffman

Funding Amount: \$216,000

Main Grant Objective: Sponsor postdoctoral fellow in development of a model plant system for understanding urban molecular ecology

\vspace{5mm}

Project Title: Drought Adaptation Within Dominant Rangeland Species: Can Genetic and Phenotypic Diversity Buffer Against Stress? (USDA 2017-67011-26072)

Dates: 01/01/2017--02/14/2019

Sponsoring Agency: USDA-NIFA-National Institute of Food and Agriculture

Principal Investigator: Melinda D. Smith

Funding Amount: \$94,999

Main Grant Objective: Sponsor predoctoral fellow stipend and research on genetic and phenotypic diversity in key prairie grass

Role: Project Director / Applicant

%----------------------------------------------------------------------------------------
%	ACADEMIC SERVICE
%----------------------------------------------------------------------------------------

\section*{ACADEMIC SERVICE}

Journal Club Founder \& Organizer, Fred Hutch Data Science Lab (2023--present

Diversity, Equity, and Inclusion (DEI) delegate, Fred Hutch Data Science Lab (2024--2025)

NIH study section reviewer, Broadening Opportunities for Computational Genomics and Data Science Education (UE5 - 3/28/24)

Executive Committee student member, Graduate Degree Program in Ecology, Colorado State University (7/1/15-6/30/16)

Faculty search committee student organizer, Department of Biology, Colorado State University (2015)

%----------------------------------------------------------------------------------------
%	PRESENTATIONS
%----------------------------------------------------------------------------------------

\section*{PRESENTATIONS}

\subsection*{Scientific Meetings}

\subsubsection*{Oral Presentations}

Ten simple rules for teaching an introduction to R, Posit::conf (upcoming 8/13/24)

Population genetics of weedy plant species in urban environments, Baltimore Ecosystem Study Annual Science Meeting (10/13/22)

Population genomics of managed and unmanaged weedy plant species in cities, Baltimore Ecosystem Study Annual Science Meeting (10/27/21)

Plastic prairie: Genetic diversity and local adaptation within blue grama grass, Ecological Society of America (8/7/18)

Dominant grasses in drylands: linking species and traits to global change in ecosystems, Front Range Student Ecology Symposium (2/14/18)

Phenotypic diversity within dominant blue grama grass across an aridity gradient, Ecological Society of America (8/8/17)

Phenotypic diversity within dominant blue grama grass across a precipitation gradient, Front Range Student Ecology Symposium (2/23/17)

Thresholds of drought response differ in a tallgrass prairie population, Ecological Society of America (8/10/16)

Genotypes of a tallgrass prairie species respond differently to drought, despite high plasticity within populations, Ecological Society of America (8/14/15)

Genotypes of a tallgrass prairie species respond differently to drought, despite high plasticity in populations, Front Range Student Ecology Symposium (2/25/15)

\subsubsection*{Poster Presentations}

Exploring Disparities in Cancer Incidence Rates Among Racial and Ethnic Groups, Annual Biomedical Research Conference for Minoritized Scientists (ABRCMS) (2023, \textit{mentee poster})

Gene expression reveals different drought response strategies in dominant dryland grass populations, Ecological Society of America (8/3-8/6/20, virtual)

Phenotypic diversity within dominant blue grama grass across an aridity gradient, Colorado State University Drought Symposium (10/11/17)

Gene expression in co-dominant prairie grasses: comparing transcriptomes using RNA-seq and de novo assembly, Ecological Society of America (8/15/14)

Composite plants: A novel method for gene screening in cowpea, Ecological Society of America (8/8/12)

\subsection*{Invited Seminars}

Nursing Grand Rounds: A.I. – Today and Tomorrow in Healthcare, Fred Hutch Cancer Center (3/5/24)

Breaking down the drought portfolio in grasslands, Vice President Office of Research Symposium (2/15/16)

\LaTeX: Introduction and tricks of the trade, GDPE Graduate Student Forum (10/7/15)

Predicting diversity: Old hypotheses, new critiques, and why you should be paying attention, GDPE Graduate Student Forum (2/18/15)

Within-population variation under drought in a tallgrass prairie species, Colorado State University Plant Supergroup (12/5/14)

%----------------------------------------------------------------------------------------
%	ADDITIONAL INFORMATION 
%----------------------------------------------------------------------------------------

\section*{ADDITIONAL INFORMATION}

\subsection*{Personal Statement}

My primary goal is to make data science and bioinformatic tools more accessible to diverse audiences by synthesizing across approaches, disciplines, and data sources. I aim to communicate across multiple domains, from academic publications to peer and community education and outreach, by developing reproducible and approachable educational resources and data science tools. I am particularly interested in empowering individuals to gain intuition for data that does not adhere to traditional norms or structure, especially in genetics research. I am also interested in integrating and developing the emerging field of environmental and ecological data science with public health initiatives. Overall, my broad, interdisciplinary background helps me approach problems holistically and serves to bring stakeholders and experts from multiple pathways together.

\subsection*{Keywords}

Genomics, data tools, data literacy, education, population genetics, gene expression, evolution, ecology

\end{document}