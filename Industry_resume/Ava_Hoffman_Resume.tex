%%%%%%%%%%%%%%%%%%%%%%%%%%%%%%%%%%%%%%%%%
% Twenty Seconds Resume/CV
% LaTeX Template
% Version 1.0 (14/7/16)
%
% Original author:
% Carmine Spagnuolo (cspagnuolo@unisa.it) with major modifications by 
% Vel (vel@LaTeXTemplates.com) and Harsh (harsh.gadgil@gmail.com)
%
% License:
% The MIT License (see included LICENSE file)
%
%%%%%%%%%%%%%%%%%%%%%%%%%%%%%%%%%%%%%%%%%

% Important note:
% Compile using LuaLaTeX, has dependencies on MiKTeX..

%----------------------------------------------------------------------------------------
%	PACKAGES AND OTHER DOCUMENT CONFIGURATIONS
%----------------------------------------------------------------------------------------

\documentclass[letterpaper]{twentysecondcv} % a4paper for A4
\usepackage{fontawesome}

% Command for printing skill overview bubbles
\newcommand\skills{ 
~
	\smartdiagram[bubble diagram]{
        \textbf{Data}\\\textbf{Science},
        \textbf{Inter-}\\\textbf{disciplinary}\\\textbf{focus},
        \textbf{Big}\\\textbf{data},
        \textbf{Data}\\\textbf{visualization},
        \textbf{Science}\\\textbf{commun-}\\\textbf{ication},
        \textbf{Project}\\\textbf{Manage-}\\\textbf{ment},
       % \textbf{Technical}\\\textbf{writing},
        \textbf{Statistical}\\\textbf{Analysis}
    }
}

% Programming skill bars
\programming{
{ Python $\textbullet$ \large \LaTeX / 0.56 }, % 20000 ~ 5.. so 2233*5/20000
{R $\textbullet$ RStan  $\textbullet$ JAGS / 5.46}} %  20000 ~ 5.. so 21844*5/20000
%R: 21844
%py: 1438
%tex: 795

% Projects text
\projects{
	\textbf{ESS 575} - Implemented Bayesian models for ecological data using JAGS\\
        \textbf{Stan Seminars} - Implemented Bayesian models for ecological data using RStan\\
        \textbf{NSCI 588} - Analyzed genomic data using Python\\
        \textbf{Overheard at ESA} - Android app developed in MIT App Builder workshop \\
        \textbf{STAT 512} - implemented more advanced principles of statistical design for research projects\\
        \textbf{STAT 511} - participation in principles of statistical design, inference, methods, \& toolbox skills for research\\
        \textbf{Teaching} - 7 semesters teaching \& 4 years student mentoring experience
        }

%----------------------------------------------------------------------------------------
%	 PERSONAL INFORMATION
%----------------------------------------------------------------------------------------
% If you don't need one or more of the below, just remove the content leaving the command, e.g. \cvnumberphone{}


\cvname{Ava Hoffman} % Your name
\cvjobtitle{ Ecologist } % Job
% title/career

\cvlinkedin{/in/ava-hoffman-0abb6054}
\cvgithub{avahoffman}
\cvnumberphone{(804) 687 7476} % Phone number
\cvsite{avahoffman.com} % Personal website
\cvmail{avamariehoffman@gmail.com} % Email address

%----------------------------------------------------------------------------------------

\begin{document}
\makeprofile % Print the sidebar

%----------------------------------------------------------------------------------------
%	 EDUCATION
%----------------------------------------------------------------------------------------
\section{Education}

\begin{twenty} % Environment for a list with descriptions
	\twentyitem
    	{2013 - 2018}
        {}
        {Ph.D, Ecology \textnormal{(GPA: 4.0/4.0)}}
        {\href{http://www.colostate.edu/}{Colorado State University, USA}}
        {}
        {}
	\twentyitem
    	{2008 - 2012}
		{}
        {B.S., Biology \textnormal{(GPA: 3.7/4.0)}}
        {\href{http://www.virginia.edu/}{University of Virginia, USA}}
        {}
        {}
\end{twenty}

\section{Research}
\begin{twenty}
	\twentyitem
    	{2013 - 2018}
		{}
        {Ph.D Candidate, USDA NIFA Predoc. Fellow}
        {\href{http://www.colostate.edu/}{Colorado State University}}
        {}
        {
       	\textbf{Dissertation}: Intraspecific diversity \& drought coping mechanisms of dominant prairie grasses
        {\begin{itemize}
        \item Awarded \$118,112 in grants to perform research
        \item Research Mentoring for Inclusivity \& Advancement in STEM Fellow, Sustainability Leadership Fellow, Vice President for Research Fellow, ESA Hackathon beginner app developer first place  {\includegraphics[scale=0.04]{img/trophy.png}}
        \item \textbf{Primary Tools}: R, RStan, shell scripts \vspace{2mm}
		\end{itemize}}
        }
\end{twenty}

\section{Recent Publications  \href{https://scholar.google.com/citations?user=k6RyLHsAAAAJ&hl=en}{{\includegraphics[scale=0.18]{img/Google_Scholar_logo_2015.png}} } }
2018. \textbf{Hoffman, AM}, et al. \href{https://peerj.com/articles/4394.pdf}{Co-dominant grasses differ in gene expression under experimental climate extremes in native tallgrass prairie.} \textit{PeerJ}.\\

2017. \textbf{Hoffman, AM} and MD Smith. \href{http://onlinelibrary.wiley.com/doi/10.1111/1755-0998.12733/full}{Gene expression differs in codominant prairie grasses under drought}. \textit{Molecular Ecology Resources}.\\


%----------------------------------------------------------------------------------------
%	 EXPERIENCE
%----------------------------------------------------------------------------------------

\section{Projects}

\begin{twenty} % Environment for a list with descriptions
    \twentyitem
   		{2018 -}
		{present}
        {\href{https://github.com/avahoffman/dryland-dominance}{Dominant species in dry ecosystems}}
       	{Colorado State University}
        {}
        {
        {\begin{itemize}
        \item Processed data from existing studies to determine the predictive power of dominant grasses in response to climate change with meta-analysis
    \end{itemize}}
        }
     \\
\twentyitem
   		{2017 -}
		{present}
        {\href{https://github.com/avahoffman/change-metabolomics}{Metabolic responses to nitrogen}}
       	{Colorado State University}
        {}
        {
        {\begin{itemize}
        \item Synthesized metabolomic, physiological, \& community responses to nitrogen using path analysis with metabolite module clustering
    \end{itemize}}
        }
     \\
\twentyitem
    	{2016 -}
		{Present}
        {\href{}{Genetic diversity in \textit{Bouteloua} grass}}
        {US Dept of Agriculture}
        {}
        {\begin{itemize}
        \item Quantified changes in the genomes of populations of this grass \& related them to differences in plant appearance \& drought strategy
        \end{itemize}}
        \\
        
	\twentyitem
    	{2014 -}
		{2018}
        {\href{https://github.com/avahoffman/nonlinear-plasticity}{Non-linear plasticity in \textit{Andropogon} grass}}
        {Colorado State University}
        {}
        {
        {\begin{itemize}
        \item Processed highly multivariate trait responses to a gradient of water availability
    \end{itemize}}
        }
    \\   
    	\twentyitem
    	{2015 -}
		{2017}
        {Gene expression (RNA) in dominant grasses}
        {Colorado State University}
        {}
        {
        {\begin{itemize}
        \item Analyzed gene expression responses of key grasses to drought using the \href{https://github.com/avahoffman/gene-expression}{\underline{de novo transcriptome assembler Trinity}}, next generation bioinformatics tools, \& \href{https://github.com/avahoffman/climate-extremes-gene-expression}{\underline{microarrays}}
    \end{itemize}}
        }
    \\   
        	\twentyitem
    	{2017}
		{}
        {\href{https://github.com/avahoffman/grasshopper-preference}{Grasshopper preference for \textit{Bouteloua}}}
        {Colorado State University}
        {}
        {
        {\begin{itemize}
        \item Modeled the responses of grasshoppers in response to different cultivars of \textit{Bouteloua} grass
    \end{itemize}}
        }
    \\   

\end{twenty}

 % printing the Last Updated text
\begin{textblock*}{4cm}(475,2) % The position of the Last Updated text on the page (75% of the page across, 2% down)
\color{mainblue}\fontsize{8pt}{10pt}\selectfont % Text styling of the text
\begin{flushright}
Last Updated \today % Last Updated text
\end{flushright}
\end{textblock*}


\end{document} 
