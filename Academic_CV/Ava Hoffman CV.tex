% Written by Ava Hoffman
% Please use as you like, but it would be nice if you credited me :)
% 
% Use LaTeX to compile and ensure style file (cv.cls) is in the same directory.
%
%%%%%%%%%%%%%%%%%%%%%%%%%%%%%%%%%%%%%%%%%

\documentclass{cv}

\begin{document}

%----------------------------------------------------------------------------------------
%	PROFESSIONAL DATA
%----------------------------------------------------------------------------------------

\name{AVA M. HOFFMAN}

\section*{PROFESSIONAL DATA}


Johns Hopkins University Krieger School of Arts \& Sciences

Department of Earth and Planetary Sciences

3400 N. Charles St

Baltimore, MD 21218

(804) 687-7476

\emailcontact{ava.hoffman@jhu.edu} $\cdot$ \emailcontact{avamariehoffman@gmail.com}

Pronouns: she/her/hers

\subsection*{Social Media}

github: avahoffman

website: \href{https://www.avahoffman.com}{https://www.avahoffman.com}

%----------------------------------------------------------------------------------------
%	EDUCATION AND TRAINING
%----------------------------------------------------------------------------------------

\section*{EDUCATION AND TRAINING}

Ph.D. / 2019 -- Colorado State University, Ecology

B.S. / 2012 -- University of Virginia, Biology \textit{with distinction}

\subsection*{Postdoctoral Training}

2020--2021 Department of Earth and Planetary Sciences,  Johns Hopkins University

%----------------------------------------------------------------------------------------
%	PROFESSIONAL EXPERIENCE
%----------------------------------------------------------------------------------------

\section*{PROFESSIONAL EXPERIENCE}

\subsection*{Johns Hopkins University}

Research Associate, Department of Biostatistics, Johns Hopkins Bloomberg School of Public Health (5/1/21--present)

Postdoctoral Fellow, Department of Earth and Planetary Sciences, Johns Hopkins Krieger School of Arts \& Sciences (3/23/20--4/30/21)

\subsection*{Other Non-JHU Professional Experience}

Data Scientist, The Boston Consulting Group (3/4/19--3/4/20), \textit{data science-based consulting role working for corporate clients}

Data Science Fellow, Insight Data Science (9/10/18--3/4/19), \textit{applied learning role building both technical products and career-focused community among fellows}

Laboratory Assistant, Biotechnology Division, DuPont-Pioneer (2012--2013)

Undergraduate Researcher, Department of Biology, University of Virginia (2011--2012)

Research Intern, United States Forest Service at Coweeta Hydrologic Laboratory LTER (2011)

Laboratory Assistant, Department of Biology, Virginia Commonwealth University (2010)

%----------------------------------------------------------------------------------------
%	PROFESSIONAL ACTIVITIES
%----------------------------------------------------------------------------------------

\section*{PROFESSIONAL ACTIVITIES}

R--Ladies Baltimore, member (2020--present)

500 Women Scientists, pod member (2016--present)

Ecological Society of America, member (2012--present)

Women in Machine Learning and Data Science (Boston), member (2019--2020)

Front Range Student Ecology Symposium, Colorado State University, abstract review committee (2015--2016)

Front Range Student Ecology Symposium, Colorado State University, executive committee (2014--2015)

Front Range Student Ecology Symposium, Colorado State University, webmaster (2014--2015)

Graduate Degree Program in Ecology, Colorado State University, Journal Club founder and member (2014)

%----------------------------------------------------------------------------------------
%	EDITORIAL AND OTHER PEER REVIEW ACTIVITIES
%----------------------------------------------------------------------------------------

\section*{EDITORIAL AND OTHER PEER REVIEW ACTIVITIES}

\subsection*{Peer Review}

Journals: Agronomy, Environmental and Experimental Botany, Global Change Biology, Journal of Ecology, New Phytologist, Plant Biology, Plant Ecology, Plants, PloS One

%----------------------------------------------------------------------------------------
%	HONORS AND AWARDS
%----------------------------------------------------------------------------------------

\section*{HONORS AND AWARDS}

\subsection*{Awards}
	
National Science Foundation Postdoctoral Research Fellowship in Biology (3/9/21, declined) %$216,000

Colorado State University Research Mentoring to Advance Inclusivity in STEM Research Grant (1/1/18) %$2500

Colorado State University School of Global Environmental Sustainability Leadership Fellowship (9/1/17) 

United States Department of Agriculture NIFA-AFRI Predoctoral Fellowship (1/1/17) %$94,999

Colorado State University Graduate Degree Program in Ecology Travel Grant (4/4/16) %$175

Colorado State University Graduate Degree Program in Ecology Research Grant (4/2/16) %$2000

City of Boulder Colorado Open Space and Mountain Parks Research Grant (3/1/2016) %$5669

Colorado State University Vice President of Research Graduate Fellow (2/29/16) %$3125

The Nature Conservancy J.E. Weaver Competitive Grant (2/10/16) %$1000

Colorado State University Department of Biology Harold Harrington Fellowship (4/10/15) %$1900

Colorado State University Department of Biology Stavros Family Fund Scholarship (4/10/15) %3021

National Science Foundation Graduate Research Fellowship Program (Honorable mention, 3/31/14) 

Colorado State University Department of Biology Sharon E. and David E. Kabes Scholarship (2014) %$590

Colorado State University Department of Biology Graduate Fellowship Award (2013) %$1000

University of Virginia Undergraduate Research Travel Grant (2012) %$1000


%2016 & \textbf{The Ecological Society of America} $\cdot$ `Hackathon' beginner app developer first place award\\
%2014-2015 & \textbf{Colorado State University} $\cdot$ Department of Biology Travel Awards (\$2258) \\

%----------------------------------------------------------------------------------------
%	PUBLICATIONS
%----------------------------------------------------------------------------------------

\section*{PUBLICATIONS}

* indicates a mentored student %or post-doctoral fellow

\subsection*{Journal Articles (Peer Reviewed)}

\begin{pubenum}

%\textbf{Hoffman, AM}, M Swall*, JA Bushey, TW Ocheltree, and MD Smith. Bouteloua Gene Expession

%\textbf{Hoffman, AM}, CC Hoffman, F Chaves, A Felton, JE Gray, W Mao, A Tatarko, L Vilonen, L Gerhardi, MD Smith. Predicting dryland ecosystem responses to global change through dominant grasses. \textit{Functional Ecology}.\\

%\textbf{Hoffman, AM}, NP Lemoine, TW Ocheltree. Metabolomic and physiological responses to nitrogen addition in dominant semi-arid grassland species. \textit{Plos One}.\\

%Bushey, JA, \textbf{AM Hoffman}, TW Ocheltree, S Gleason, MD Smith. Water limitation reveals local adaptation and plasticity in the drought tolerance strategies of \textit{Bouteloua gracilis}. \textit{Environmental and Experimental Botany}. \\

\item Carroll, CJW, IJ Slette, RJ Griffin-Nolan, LE Baur, \textbf{AM Hoffman}, EM Denton, JE Gray, AK Post, MK Johnston, Q Yu, SL Collins, Y Luo, MD Smith, AK Knapp (2021) Is a drought a drought in grasslands? Productivity responses to different types of drought. \textit{Oecologia}. \doi{10.1007/s00442-020-04793-8}

\item\textbf{Hoffman, AM} and MD Smith (2020) Nonlinear drought plasticity reveals intraspecific diversity in a dominant grass species. \textit{Functional Ecology}. \doi{10.1111/1365-2435.13713}

\item Knapp, AK, A Chen, RJ Griffin-Nolan, LE Baur, CJW Carroll, JE Gray, \textbf{AM Hoffman}, X Li, AK Post, IJ Slette, SL Collins, Y Luo, MD Smith (2020) Resolving the Dust Bowl paradox of grassland responses to extreme drought. \textit{PNAS}. \doi{10.1073/pnas.1922030117}

\item\textbf{Hoffman, AM}, JA Bushey, TW Ocheltree, MD Smith (2020) Genetic and functional variation across regional and local scales is associated with climate in a foundational prairie grass. \textit{New Phytologist}. \doi{10.1111/nph.16547}

\item Wilcox, KR, SE Koerner, DL Hoover, AK Borkenhagen, DE Burkepile, SL Collins, \textbf{AM Hoffman}, KP Kirkman, AK Knapp, T Strydom, DI Thompson, and MD Smith (2020) Rapid recovery of ecosystem function following extreme drought in a South African savanna-grassland. \textit{Ecology}. \doi{10.1002/ecy.2983}

\item\textbf{Hoffman, AM}, H Perretta*, NP Lemoine, and MD Smith (2019) Blue grama grass genotype affects palatability and preference by semi-arid steppe grasshoppers. \textit{Acta Oecologia}. \doi{10.1016/j.actao.2019.03.001}

\item Griffin-Nolan, RJ, D Blumenthal, S Collins, T Farkas, \textbf{AM Hoffman}, K Mueller, TW Ocheltree, MD Smith, AK Knapp (2019) Shifts in plant functional composition following long-term drought in grasslands. \textit{Journal of Ecology}. \doi{10.1111/1365-2745.13252}

\item Smith, MD, SE Koerner, AK Knapp, ML Avolio, FA Chaves, EM Denton, J Dietrich, DJ Gibson, J Gray, \textbf{AM Hoffman}, DL Hoover, KJ Komatsu, A Silletti, KR Wilcox, Q Yu, and JM Blair (2019) Mass ratio effects underlie ecosystem responses to environmental change. \textit{Journal of Ecology}. \doi{10.1111/1365-2745.13330}

\item\textbf{Hoffman, AM} and MD Smith (2018) Thinking inside the box: Tissue culture for plant propagation in a key ecological species, \textit{Andropogon gerardii}. \textit{American Journal of Plant Sciences}. \doi{10.4236/ajps.2018.910144} 

\item Knapp, AK, C Carroll, RJ Griffin-Nolan, IJ Slette, F Chaves, L Baur, AJ Felton, JE Gray, \textbf{AM Hoffman}, NP Lemoine, W Mao, A Post, MD Smith (2018) A reality check for climate change experiments: do they reflect the real world? \textit{Ecology}. \doi{10.1002/ecy.2474}

\item Griffin-Nolan, RJ, JA Bushey, CJW Carroll, A Challis, J Chieppa, M Garbowski, \textbf{AM Hoffman}, AK Post, IJ Slette, D Spitzer, D Zambonini, TW Ocheltree, DT Tissue, AK Knapp  (2018) Trait selection and community weighting are key to understanding ecosystem responses to changing precipitation regimes. \textit{Functional Ecology}. \doi{10.1111/1365-2435.13135}

\item\textbf{Hoffman, AM}, ML Avolio, AK Knapp, MD Smith (2018) Co-dominant grasses differ in gene expression under experimental climate extremes in native tallgrass prairie. \textit{PeerJ}. \doi{10.7717/peerj.4394}

\item\textbf{Hoffman, AM} and MD Smith (2017) Gene expression differs in codominant prairie grasses under drought. \textit{Molecular Ecology Resources}. \doi{10.1111/1755-0998.12733}

\item Avolio, ML, \textbf{AM Hoffman}, MD Smith (2017) Linking gene regulation, physiology, and plant biomass allocation in \textit{Andropogon gerardii} in response to drought. \textit{Plant Ecology}. \doi{10.1007/s11258-017-0773-3}

\item Lemoine, NP, \textbf{AM Hoffman}, A Felton, L Baur, F Chaves, J Gray, Q Yu, MD Smith (2016) Underappreciated problems of low statistical power in ecological field studies. \textit{Ecology}. \doi{10.1002/ecy.1506}

\item Smith, MD, \textbf{AM Hoffman}, ML Avolio (2016) Gene expression patterns of two dominant tallgrass prairie species differ in response to warming and altered precipitation. \textit{Scientific Reports}. \doi{10.1038/srep25522}

\item Mellor, KE, \textbf{AM Hoffman}, MP Timko (2012) Use of ex vitro composite plants to study the interaction of cowpea (\textit{Vigna unguiculata} L.) with the root parasitic angiosperm \textit{Striga gesnerioides}. \textit{Plant Methods}. \doi{10.1186/1746-4811-8-22}

\sloppy % break url onto next line
\item\textbf{Hoffman, AM} (2012) Estimating tree transpiration accurately depends on wood type and species: a study of four southern Appalachian tree species. \textit{The Oculus}. \href{https://issuu.com/theoculus/docs/spring2012}{https://issuu.com/theoculus/docs/spring2012}

\item Zinnert, JC, JD Nelson, \textbf{AM Hoffman} (2011) Effects of salinity on physiological responses and the photochemical reflectance index in two co-occurring coastal shrubs. \textit{Plant \& Soil}. \doi{10.1007/s11104-011-0955-z}

\end{pubenum}

\subsection*{Articles, Editorials and Other Publications Not Peer Reviewed}

Zinnert, JC, JD Nelson, JK Vick, \textbf{AM Hoffman}, DR Young (2010) Rethinking chlorophyll responses to stress: Fluorescence and reflectance remote sensing in a coastal environment. Proceedings of the 4th International Workshop on Remote Sensing of Vegetation Fluorescence, Valencia, Spain.

%----------------------------------------------------------------------------------------
%	PRACTICE ACTIVITIES
%----------------------------------------------------------------------------------------

\section*{PRACTICE ACTIVITIES}

\subsection*{Presentations to Policymakers, Communities, and Other Stakeholders}

Front Range Open Space Research Symposium (Boulder, CO, 4/11/17): Phenotypic diversity within dominant blue grama grass across an aridity gradient

%----------------------------------------------------------------------------------------
% Break page and add Part II header
\newpage
\name{AVA M. HOFFMAN}
\parttwo
%----------------------------------------------------------------------------------------

%----------------------------------------------------------------------------------------
%	TEACHING
%----------------------------------------------------------------------------------------

\section*{TEACHING}


\subsection*{Capstone Advisees}

Robotham, Daniel J., B.S. in Biological Sciences with Honors, Colorado State University (presented 5/15/15)

\hangpara{5pt}{0}Thesis Title: Determining the effects of water stress on co-occurring native \textit{Andropogon gerardii} and exotic (\textit{Bothriochloa bladhii}) C4 grasses

\subsection*{Research Advisees}

Swall, Madeleine, Undergraduate Student Researcher and Research Mentoring to Advance Inclusivity in STEM mentee, Colorado State University (2018)

Perretta, Holly, Undergraduate REU Student Researcher, Colorado State University (2016--2017)

Lock, Abigail, Undergraduate Research Assistant, Colorado State University (2017)

Gaudrealt, Brigitte, Undergraduate Research Assistant, Colorado State University (2016)

Magbual, Brianna, Undergraduate Research Assistant, Colorado State University (2014--2015)

Brown, Destiny, Undergraduate Research Assistant, Colorado State University (2014)

\subsection*{Classroom Instruction -- Instructor of Record}

\subsubsection*{Johns Hopkins University}

Baltimore Community Data Science (Special Topics Course, Spring 2022)

Introduction to \texttt{R} for Public Health Researchers (Winter 2022, enrollment: 30)

Introduction to \texttt{R}  for Public Health Researchers (Summer 2021, enrollment: 33)

\subsection*{Classroom Instruction -- Teaching Fellow / Assistant}

\subsubsection*{Colorado State University}

Molecular and General Genetics (BZ 350), Recitation instructor (Fall 2016)

Community Ecology (ECOL 600), Recitation instructor (Spring 2016)

Foundations of Ecology (ECOL 505), Recitation instructor with class lectures (Fall 2015)

Molecular and General Genetics (BZ 350), Recitation instructor (Spring 2015)

Molecular and General Genetics (BZ 350), Recitation instructor (Fall 2014)

\subsubsection*{University of Virginia}

Organic Chemistry Lab II (CHEM 2421), Laboratory instructor (Spring 2011)

Organic Chemistry Lab I (CHEM 2411), Laboratory instructor (Fall 2010)


\subsection*{Other Teaching}

Data Wrangling with R (3 day course, University of Washington Summer Institute, 2021)

Data Visualization using R and ggplot (guest lecture, Colorado State University, 2016)

%----------------------------------------------------------------------------------------
%	RESEARCH GRANT PARTICIPATION
%----------------------------------------------------------------------------------------

\section*{RESEARCH GRANT PARTICIPATION}

\subsection*{Current Support}

Project Title: EVO-LTER: Leveraging long-term ecological research in grasslands: facilitating collaborations between ecologists and evolutionary biologists (DEB-2110351)

Dates: 3/2022--2/2024 (current)

Sponsoring Agency: National Science Foundation

Principal Investigator: Meghan L. Avolio

Funding Amount: \$99,933

Main Grant Objective: Facilitate collaborative workshop to foster more evolutionary and molecular work at NSF Long Term Ecological Research sites

Role: Co-PI

\subsection*{Past Support}

Project Title: Collaborative Research: MSB-FRA: Alternative futures for the American residential macro systems (DEB-1836034)

Dates: 01/2017--12/2020

Sponsoring Agency: National Science Foundation

Principal Investigator: Meghan L. Avolio

Funding Amount: \$90,963

Main Grant Objective: Reveal genomic population structure and phenotypic differences among urban weeds in North American cities

Role: Postdoctoral fellow / Project Manager

\vspace{3mm}

Project Title: CityCress: A plant model for genetic, phenotypic, and gene regulatory relationships in urban environments (NSF-2109727)

Dates: Award declined by PI

Sponsoring Agency: National Science Foundation

Principal Investigator: Ava M. Hoffman

Funding Amount: \$216,000

Main Grant Objective: Sponsor postdoctoral fellow in development of a model plant system for understanding urban molecular ecology

\vspace{3mm}

Project Title: Drought Adaptation Within Dominant Rangeland Species: Can Genetic and Phenotypic Diversity Buffer Against Stress? (USDA 2017-67011-26072)

Dates: 01/01/2017-02/14/2019

Sponsoring Agency: USDA-NIFA-National Institute of Food and Agriculture

Principal Investigator: Melinda D. Smith

Funding Amount: \$94,999

Main Grant Objective: Sponsor predoctoral fellow stipend and research on genetic and phenotypic diversity in key prairie grass

Role: Project Director / Applicant

%edge project.. ?

%----------------------------------------------------------------------------------------
%	ACADEMIC SERVICE
%----------------------------------------------------------------------------------------

\section*{ACADEMIC SERVICE}

Executive Committee student member, Graduate Degree Program in Ecology, Colorado State University (7/1/15-6/30/16)

Faculty search committee student organizer, Department of Biology, Colorado State University (2015)

%----------------------------------------------------------------------------------------
%	PRESENTATIONS
%----------------------------------------------------------------------------------------

\section*{PRESENTATIONS}

\subsection*{Scientific Meetings}

\subsubsection*{Oral Presentations}

Population genomics of managed and unmanaged weedy plant species in cities, Baltimore Ecosystem Study Annual Science Meeting (10/27/21)

Plastic prairie: Genetic diversity and local adaptation within blue grama grass, Ecological Society of America (8/7/18)

Dominant grasses in drylands: linking species and traits to global change in ecosystems, Front Range Student Ecology Symposium (2/14/18)

Phenotypic diversity within dominant blue grama grass across an aridity gradient, Ecological Society of America (8/8/17)

Phenotypic diversity within dominant blue grama grass across a precipitation gradient, Front Range Student Ecology Symposium (2/23/17)

Thresholds of drought response differ in a tallgrass prairie population, Ecological Society of America (8/10/16)

Genotypes of a tallgrass prairie species respond differently to drought, despite high plasticity within populations, Ecological Society of America (8/14/15)

Genotypes of a tallgrass prairie species respond differently to drought, despite high plasticity in populations, Front Range Student Ecology Symposium (2/25/15)

\subsubsection*{Poster Presentations}

Gene expression reveals different drought response strategies in dominant dryland grass populations, Ecological Society of America (8/3-8/6/20, virtual)

Phenotypic diversity within dominant blue grama grass across an aridity gradient, Colorado State University Drought Symposium (10/11/17)

Gene expression in co-dominant prairie grasses: comparing transcriptomes using RNA-seq and de novo assembly, Ecological Society of America (8/15/14)

Composite plants: A novel method for gene screening in cowpea, Ecological Society of America (8/8/12)

\subsection*{Invited Seminars}

Breaking down the drought portfolio in grasslands, Vice President Office of Research Symposium (2/15/16)

\LaTeX: Introduction and tricks of the trade, GDPE Graduate Student Forum (10/7/15)

Predicting diversity: Old hypotheses, new critiques, and why you should be paying attention, GDPE Graduate Student Forum (2/18/15)

Within-population variation under drought in a tallgrass prairie species, Colorado State University Plant Supergroup (12/5/14)

%----------------------------------------------------------------------------------------
%	ADDITIONAL INFORMATION 
%----------------------------------------------------------------------------------------

\section*{ADDITIONAL INFORMATION}

\subsection*{Personal Statement}

My primary goal is to make data science and bioinformatic tools more accessible to diverse audiences by synthesizing across approaches, disciplines, and data sources. I aim to communicate across multiple domains, from academic publications to peer and community education and outreach, by developing reproducible and approachable educational resources and data science tools. I am particularly interested in empowering individuals to gain intuition for data that does not adhere to traditional norms or structure, especially in genetics research. I am also interested in integrating and developing the emerging field of environmental and ecological data science with public health initiatives. Overall, my broad, interdisciplinary background helps me approach problems holistically and serves to bring stakeholders and experts from multiple pathways together.

\subsection*{Keywords}

Genomics, data tools, data literacy, education, population genetics, gene expression, evolution, ecology

\end{document}